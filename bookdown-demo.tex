\documentclass[]{book}
\usepackage{lmodern}
\usepackage{amssymb,amsmath}
\usepackage{ifxetex,ifluatex}
\usepackage{fixltx2e} % provides \textsubscript
\ifnum 0\ifxetex 1\fi\ifluatex 1\fi=0 % if pdftex
  \usepackage[T1]{fontenc}
  \usepackage[utf8]{inputenc}
\else % if luatex or xelatex
  \ifxetex
    \usepackage{mathspec}
  \else
    \usepackage{fontspec}
  \fi
  \defaultfontfeatures{Ligatures=TeX,Scale=MatchLowercase}
\fi
% use upquote if available, for straight quotes in verbatim environments
\IfFileExists{upquote.sty}{\usepackage{upquote}}{}
% use microtype if available
\IfFileExists{microtype.sty}{%
\usepackage{microtype}
\UseMicrotypeSet[protrusion]{basicmath} % disable protrusion for tt fonts
}{}
\usepackage{hyperref}
\hypersetup{unicode=true,
            pdftitle={Competition Policy},
            pdfauthor={Marco Alberti, Erik Brouwer, Clemens Fiedler and Misja Mikkers},
            pdfborder={0 0 0},
            breaklinks=true}
\urlstyle{same}  % don't use monospace font for urls
\usepackage{natbib}
\bibliographystyle{apalike}
\usepackage{longtable,booktabs}
\usepackage{graphicx,grffile}
\makeatletter
\def\maxwidth{\ifdim\Gin@nat@width>\linewidth\linewidth\else\Gin@nat@width\fi}
\def\maxheight{\ifdim\Gin@nat@height>\textheight\textheight\else\Gin@nat@height\fi}
\makeatother
% Scale images if necessary, so that they will not overflow the page
% margins by default, and it is still possible to overwrite the defaults
% using explicit options in \includegraphics[width, height, ...]{}
\setkeys{Gin}{width=\maxwidth,height=\maxheight,keepaspectratio}
\IfFileExists{parskip.sty}{%
\usepackage{parskip}
}{% else
\setlength{\parindent}{0pt}
\setlength{\parskip}{6pt plus 2pt minus 1pt}
}
\setlength{\emergencystretch}{3em}  % prevent overfull lines
\providecommand{\tightlist}{%
  \setlength{\itemsep}{0pt}\setlength{\parskip}{0pt}}
\setcounter{secnumdepth}{5}
% Redefines (sub)paragraphs to behave more like sections
\ifx\paragraph\undefined\else
\let\oldparagraph\paragraph
\renewcommand{\paragraph}[1]{\oldparagraph{#1}\mbox{}}
\fi
\ifx\subparagraph\undefined\else
\let\oldsubparagraph\subparagraph
\renewcommand{\subparagraph}[1]{\oldsubparagraph{#1}\mbox{}}
\fi

%%% Use protect on footnotes to avoid problems with footnotes in titles
\let\rmarkdownfootnote\footnote%
\def\footnote{\protect\rmarkdownfootnote}

%%% Change title format to be more compact
\usepackage{titling}

% Create subtitle command for use in maketitle
\providecommand{\subtitle}[1]{
  \posttitle{
    \begin{center}\large#1\end{center}
    }
}

\setlength{\droptitle}{-2em}

  \title{Competition Policy}
    \pretitle{\vspace{\droptitle}\centering\huge}
  \posttitle{\par}
    \author{Marco Alberti, Erik Brouwer, Clemens Fiedler and Misja Mikkers}
    \preauthor{\centering\large\emph}
  \postauthor{\par}
      \predate{\centering\large\emph}
  \postdate{\par}
    \date{2020-02-12}

\usepackage{booktabs}
\usepackage{amsthm}
\makeatletter
\def\thm@space@setup{%
  \thm@preskip=8pt plus 2pt minus 4pt
  \thm@postskip=\thm@preskip
}
\makeatother

\begin{document}
\maketitle

{
\setcounter{tocdepth}{1}
\tableofcontents
}
\chapter{Prerequisites}\label{prerequisites}

No prerequisites required.

\chapter{Introduction}\label{intro}

\section{Team}\label{team}

In 2019-2020 this course is taught by:

\begin{itemize}
\tightlist
\item
  Santiago Bohorquez
\item
  Jose Carreno Bustos
\item
  Misja Mikkers
\item
  Marius-Lucian Prisacuta
\item
  Florian Sniekers
\item
  Jierui Yang.
\end{itemize}

\section{Datacamp}\label{datacamp}

We are very happy that we partner with datacamp for this course to teach
you both python and R.

Datacamp offers great on-line courses for you to learn R and python.

\section{Important}\label{important}

Things to do if you want to follow this course:

\begin{itemize}
\tightlist
\item
  go to the Canvas page of this course.
\item
  click on the link to get personal access to datacamp, and enroll with
  your \emph{Tilburg University email address}
\item
  go to the russet server at \url{https://russet.uvt.nl}
\item
  log into the server and click on the green button ``Start My Server''
\item
  copy the address from the address field in your browser (you need to
  paste this in a webform)
\item
  after you have done this, go to the webform:
  \url{https://forms.gle/iZG5AaM3VpnBdGQa9}
\item
  and fill in this webform.
\item
  note that you need to fill in the webform on or before \emph{January
  31}
\item
  If you do not fill in the webform before the deadline, you cannot get
  a grade for this course. We use the webform also to plan the
  tutorials, so keep an eye on Canvas before your first tutorial
\end{itemize}

\section{Questions}\label{questions}

There are no stupid questions, it's stupid not to ask questions. We
encourage you to post your questions in the discussion section on
Canvas.

Only when you need to include privately sensitive information (``my cat
has passed away''), you can send an email. Always provide us with the
following information: - say whether you are an ECO or EBE student -
mention the group number of your tutorial and/or the name of your
tutorial teacher - explain your question

\chapter{Schedule}\label{schedule}

\section{Lecture 1 Introduction 30-01
(MM/CF)}\label{lecture-1-introduction-30-01-mmcf}

\subsection{In class:}\label{in-class}

organization of the course rules of the game basics Markdown basics
python basics R To prepare before class:

Install Anaconda, R and R studio and bring your laptop 3.2 Lecture 2
Competition law and its objectives 01-02 (MM) In class:

classroom experiment competition presentation about the role of
competition and regulation

\subsection{To prepare before class:}\label{to-prepare-before-class}

Motta, Chapter 1. European Commission (2014). The European Union
explained: Competition. Making markets work better.
(\url{http://ec.europa.eu/competition/publications/}) 3.3 Lecture 3
Market definition and the measurement of competition and market power
06-02 (MA/RC) In class:

\section{Estimation of market power}\label{estimation-of-market-power}

\subsection{In class}\label{in-class-1}

guest lecture by Ramsis Croes (NZa/Erasmus Univeristy) demonstration of
a logit demand estimation

\subsection{To prepare before class:}\label{to-prepare-before-class-1}

Motta, Chapters 2 and 3. Gaynor: Paper will be uploaded to students

\section{Lecture 4 Collusion and Horizontal Agreements 08-02
(MM)}\label{lecture-4-collusion-and-horizontal-agreements-08-02-mm}

\subsection{In class:}\label{in-class-2}

classroom experiment with strategic interaction duopolies presentation
about collusion

\subsection{To prepare:}\label{to-prepare}

Motta, Chapter 4 (with the exception of 4.3)

\section{Lecture 5 Continuation of Lecture 2, deterrence and cartels
13-02
(CF)}\label{lecture-5-continuation-of-lecture-2-deterrence-and-cartels-13-02-cf}

\subsection{In class:}\label{in-class-3}

theory

\subsection{To prepare before class:}\label{to-prepare-before-class-2}

Motta, Chapter 4 (with the exception of 4.3)

\section{Lecture 6 Article 102 TFEU: Abuse of a dominant position 14-02
(MA/AB)}\label{lecture-6-article-102-tfeu-abuse-of-a-dominant-position-14-02-maab}

In class:

\begin{itemize}
\tightlist
\item
  lecture
\end{itemize}

To prepare before class:

\begin{itemize}
\tightlist
\item
  Motta, chapter 7
\item
  European Commission: Guidance on enforcement priorities in applying
  Article 102 of the EC Treaty to abusive exclusionary conduct by
  dominant undertakings, 2009
\end{itemize}

\section{Lecture 7 Mergers 19-02 (MA)}\label{lecture-7-mergers-19-02-ma}

In class:

\begin{itemize}
\tightlist
\item
  lecture
\item
  guest lecture from ACM about merger assessment
\end{itemize}

To prepare before class :

\begin{itemize}
\tightlist
\item
  Motta, Chapter 5 (with the exception of 5.4) and chapter 6 (6.2.4,
  6.4.2, 6.6) (EB)
\item
  European Commission: Guidelines on the assessment of horizontal
  mergers, 2004
\item
  European Commission: Guidelines on the assessment of non-horizontal
  mergers, 2008
\end{itemize}

\section{Lecture 8 Leniency schemes and bid rigging 21-02
(CF)}\label{lecture-8-leniency-schemes-and-bid-rigging-21-02-cf}

In class

\begin{itemize}
\tightlist
\item
  Lecture
\end{itemize}

Prepare before class:

\begin{itemize}
\tightlist
\item
  \url{https://www.researchgate.net/publication/300700468_Exploitation_and_Induced_Tacit_Collusion_A_Classroom_Experiment_of_Corporate_Leniency_Programs}
\item
  Bigoni, M., Fridolfson, S., Le Coq, C. and G. Spagnolo (2012). Fines,
  leniency and rewards in antitrust. The Rand Journal of Economics 43
  (2), 368-390.
\item
  McAfee, R. P., \& McMillan, J. (1992). Auctions and Bidding Rings.
  American Economic Review, 82(3), 579-599.
\end{itemize}

\section{Lecture 9 Horizontal mergers - part 2 04-03
(MA)}\label{lecture-9-horizontal-mergers---part-2-04-03-ma}

In class:

\begin{itemize}
\tightlist
\item
  Lecture
\end{itemize}

To prepare before class:

\begin{itemize}
\tightlist
\item
  Farrell, J., \& Shapiro, C. (1990). Horizontal mergers: an equilibrium
  analysis. The American Economic Review, 80 (1), 107-126
\item
  Federico, G., Langus, G. and T. Valletti (2017). Horizontal mergers
  and product innovation: an economic framework. UPF Working paper
  2017-1579.
  \href{https://papers.ssrn.com/sol3/papers.cfm?abstract_id=2999178M}{Available
  at}
\item
  Ivaldi \& F. Verboven (2005): Quantifying the effects from horizontal
  mergers in European competition policy
  \href{https://ideas.repec.org/p/cpr/ceprdp/2697.html}{to download}
\end{itemize}

\section{Lecture 10 Price regulation 06-03
(MM/BP)}\label{lecture-10-price-regulation-06-03-mmbp}

In class

\begin{itemize}
\tightlist
\item
  Presentation about the theory of regulation
\item
  Presentation from Bas Postema (ACM) about the regulation of the Dutch
  electricity networks
\end{itemize}

To prepare

\begin{itemize}
\item
  P. Agrell and P. Bogetoft (2004) Evolutionary Regulation: From CPI-X
  towards Contestability
  \href{https://www.sumicsid.com/reg/papers/encore.pdf}{download}
\item
  ACM (2017) Incentive regulation of the gas and electricity networks in
  the Netherlands
  \href{https://www.acm.nl/sites/default/files/old_publication/publicaties/17231_incentive-regulation-of-the-gas-and-elektricity-networks-in-the-netherlands-2017-05-17.pdf}{download}
\end{itemize}

\section{Lecture 11 Benchmarking 11-03
(MM/VS)}\label{lecture-11-benchmarking-11-03-mmvs}

In class

\begin{itemize}
\tightlist
\item
  Presentation from Victoria Shestalova (NZa) about DEA
\item
  DEA in R
\end{itemize}

To prepare:

\begin{itemize}
\tightlist
\item
  P. Agrell and P. Bogetoft (2002), DEA-Based incentive regimes in
  health care provision
  \href{https://www.researchgate.net/profile/Peter_Bogetoft/publication/23515165_DEA-Based_Incentive_Regimes_in_Health-Care_Provision/links/0912f50ae2f961fe8b000000.pdf}{to
  download}
\item
  A. Arcos-Vargas, F. Núñez-Hernández, Gabriel Villa-Caro, (2017), A DEA
  analysis of electricity distribution in Spain: An industrial policy
  recommendation, Energy Policy 102 (583 - 592)
\end{itemize}

\section{Lecture 12 Vertical integration 13-03
(MM/KK)}\label{lecture-12-vertical-integration-13-03-mmkk}

In class:

\begin{itemize}
\tightlist
\item
  classroom experiment vertical mergers
\item
  presentation about vertical mergers and vertical restraints by Katalin
  Katona (NZa/Tilburg University)
\end{itemize}

To prepare before class:

\begin{itemize}
\tightlist
\item
  Motta, Chapter 6 (with the exception of the two starred (**) sections)
\end{itemize}

\section{Lecture 13 Abuse of a dominant position: predatory pricing
18-03
(MA/MM)}\label{lecture-13-abuse-of-a-dominant-position-predatory-pricing-18-03-mamm}

In class:

\begin{itemize}
\tightlist
\item
  classroom experiment
\item
  lecture
\end{itemize}

To prepare before class:

\begin{itemize}
\tightlist
\item
  Goolsbee, A., Syverson, C. (2008). How do incumbents respond to the
  threat of entry? Evidence from the major airline. The Quartely Journal
  of Economics, Vol. 123, Issue 4, 1611-1633
  \url{https://academic.oup.com/qje/article/123/4/1611/1933206}
\item
  Nurski, L., Verboven. F. (2016). Exclusive Dealing as a Barrier to
  Entry? Evidence from Automobiles. The review of economic studies, Vol
  83, Issue 3, 1156-1188.,
  \url{https://academic.oup.com/restud/article/83/3/1156/2461273}
\end{itemize}

\section{Problems}\label{problems}

The idea with these lectures is that students in teams will present
solutions to policy makers for some competition and regulation problems.
The students will present the case in class to the government.

the presentation (20 mins) should:

\begin{itemize}
\tightlist
\item
  describe the problem
\item
  discuss the relevant academic literature
\item
  present policy options (including pro's and cons from an economic
  perspective)
\end{itemize}

\section{Lecture 14 Transparancy of prices in the hospital market 20-03
(MM)}\label{lecture-14-transparancy-of-prices-in-the-hospital-market-20-03-mm}

\begin{itemize}
\tightlist
\item
  Teams A, B, C, D
\end{itemize}

\section{Lecture 15 Competition problem number 2 25-03
(MA)}\label{lecture-15-competition-problem-number-2-25-03-ma}

\begin{itemize}
\tightlist
\item
  teams E, F, G, H
\end{itemize}

\section{Lecture 16 Regulation problem 27-03
(MM)}\label{lecture-16-regulation-problem-27-03-mm}

\begin{itemize}
\tightlist
\item
  teams I, J, K, L
\end{itemize}

\section{Cases}\label{cases}

In the cases students will form new teams. Some teams will represent the
government (Comptetition Authority or Regulator), the other teams will
represent the firms. We will provide some data, so the teams can also do
a quantative analysis.

The students not assigned to a team will form the jury and will have to
formulate a verdict.

\section{Lecture 17 Competition case 01-04
(MA)}\label{lecture-17-competition-case-01-04-ma}

\begin{enumerate}
\def\labelenumi{\alph{enumi}.}
\tightlist
\item
  Competition Authority: teams 1 and 2
\item
  Firms: teams 3 and 4
\end{enumerate}

\section{Lecture 18 Regulation case 03-04
(MM)}\label{lecture-18-regulation-case-03-04-mm}

Note: Nursing homes

\begin{enumerate}
\def\labelenumi{\alph{enumi}.}
\item
  Regulator: teams 5 and 6
\item ~
  \chapter{Firms: teams 7 and 8}\label{firms-teams-7-and-8}

  Finally, we urge you to use google (or other search engines like
  DuckDuckGo) and stackoverflow with your assignments. Some students
  find this weird at the beginning: should we not teach you everything
  that you need to know? The answer is no for a number of reasons.
  First, even professional programmers use google and stackoverflow all
  the time. If you are on Quora; see this post and this one. Second,
  python and R are open source and lots of people work with it. If you
  encounter a problem, chances are that someone else had the same
  problem and knows the solution to it. There is not need to ``invent
  the wheel''. Use the resources available to you. If you copy a lot of
  code, you should add a reference. Finally, because python and R are
  open source, they develop rapidly. The things that we teach you now,
  will be obsolete in a couple of years time. Hence, you need to be able
  to find your way around also in 10 years time. To start practicing
  this, use google now.
  \textgreater{}\textgreater{}\textgreater{}\textgreater{}\textgreater{}\textgreater{}\textgreater{}
  54edd162eaf857e745bc2b1bfdd68064d89af608
\end{enumerate}

The only warning here is: at the exam you will not have access to the
whole internet. So, also make sure that you can find help in the jupyter
notebook. We will practice this in class.

\chapter{Important}\label{important-1}

\emph{Things to do if you want to follow this course:}

\begin{itemize}
\tightlist
\item
  go to the Canvas page of this course.
\item
  click on the link to get personal access to datacamp, and enroll with
  your \emph{Tilburg University email address}
\item
  go to the russet server at \url{https://russet.uvt.nl}
\item
  log into the server and click on the green button ``Start My Server''
\item
  copy the address from the address field in your browser (you need to
  paste this in a webform)
\item
  after you have done this, go to the webform:
  \url{https://forms.gle/iZG5AaM3VpnBdGQa9}
\item
  and fill in this webform.
\item
  note that you need to fill in the webform on or before \emph{January
  31}
\item
  if you do not fill in the webform before the deadline, you cannot get
  a grade for this course
\item
  we use the webform also to plan the tutorials, so keep an eye on
  Canvas before your first tutorial
\end{itemize}

\chapter{Information first Lecture}\label{information-first-lecture}

\section{Introduction Programming}\label{introduction-programming}

Misja Mikkers \& Florian Sniekers

\section{Table of Contents}\label{table-of-contents}

\begin{itemize}
\tightlist
\item
  Introduction
\item
  markdown
\item
  Second part: how to make sure you get the course material and a grade
  for this course
\end{itemize}

\section{Introduction}\label{introduction}

\begin{itemize}
\tightlist
\item
  Don't panic
\item
  this is a programming course
\item
  we know that many of you are not too keen on computers (beyond MS
  Office)
\item
  this will be a gentle introduction to open source software
\item
  it will not become too sophisticated
\item
  it is meant for everyone to understand
\item
  especially, if you never did any programming before
\item
  Why this course?
\item
  mainly to teach you to use your computer better
\item
  to be able to use open source (``free'') software
\item
  to solve problems together with readable documentation on how you
  solved it (``reproducible research'')
\item
  on this last point, office products like excel score rather badly
\item
  you will use R and python in courses in the years to come
\end{itemize}

\subsection{Who teaches this course?}\label{who-teaches-this-course}

In 2019-2020 this course is taught by:

\begin{itemize}
\tightlist
\item
  Santiago Bohorquez
\item
  Jose Carreno Bustos
\item
  Misja Mikkers
\item
  Marius-Lucian Prisacuta
\item
  Florian Sniekers
\item
  Jierui Yang.
\end{itemize}

\subsection{How do we teach this
course?}\label{how-do-we-teach-this-course}

\begin{itemize}
\tightlist
\item
  online lectures on Datacamp
\item
  tutorials: with plenty of time to ask questions
\item
  there are a number of ``regular'' tutorials and one in the computer
  lab
\item
  if you do not have a laptop, attend the tutorial in the computer lab
\item
  if you do have a laptop, attend your regular tutorial group
\item
  no need to attend both!
\item
  we may drop some tutorial groups, so check Canvas!
\item
  schedule can be found in the chapter schedule on this website
\item
  we can track your progress on datacamp
\item
  assignment notebooks to be made \emph{before the class}
\item
  class notebooks that we do together \emph{in class} (to allow you to
  ask questions)
\end{itemize}

\subsection{Information about the
course}\label{information-about-the-course}

\begin{itemize}
\item
  all information about the course can be found on this website
\item
  pay attention to: -the schedule: explaining when you need to do what
  -the rules for the exam explaining how the exam works and a practice
  exam
\item
  Your grade -There are two separate ways to earn your grade
\item
  regular route: -midterm on python -end of semester exam on R
\item
  resit -exam on python and R combined -you cannot use grades from one
  route for the other one.
\end{itemize}

\subsection{Exam}\label{exam}

\begin{itemize}
\tightlist
\item
  for more information see the exam chapter
\item
  check the instructions for the exam
\item
  do not open your exam file after you have finished
\item
  if you do, your exam will not be graded (even if you did not change
  anything)
\item
  at the exam you can freely copy and paste from the assignments we did
  in class
\item
  we will not post answers to the assignments
\item
  make sure you attend the tutorials and pay attention in class!
\end{itemize}

\subsection{Datacamp}\label{datacamp-1}

\begin{itemize}
\tightlist
\item
  you need to sign up for Datacamp!
\item
  for details see the chapter Important
\item
  note the deadline for filling in the webform!
\item
  if you miss the deadline, you may have to pay for premium content on
  Datacamp yourself
\end{itemize}

\subsection{markdown}\label{markdown}

\begin{itemize}
\item
  syntax
\item
  markdown allows you to create structure in a simple way
\item
  examples are: \texttt{\#\ this\ is\ a\ heading}

  \texttt{\#\#\ subheading}

  \texttt{*\ first\ bullet} \texttt{*\ second\ bullet}

  \texttt{{[}link\ text{]}(actual\ link,\ e.g.\ http://www.etc)}

  \texttt{!{[}Alt\ text\ for\ image{]}(/path/to/img.jpg\ "Optional\ title")}
\item
  look on the web for other syntax like footnotes etc.
\item
  equations you can type in latex
\item
  latex is great word processing software for now, we only need it to
  write math you can guess what the following will do:
\end{itemize}

\texttt{\$x\^{}2\$,\ \$\textbackslash{}beta\$,\ \$\textbackslash{}sqrt\{9\}\$,\ \$\textbackslash{}frac\{1\}\{2\}\$,\ \$\textbackslash{}bar\ x\$}

\texttt{\textbackslash{}begin\{equation\}}

\texttt{a\^{}2\ +\ b\^{}2\ =\ c\^{}2}

\texttt{\textbackslash{}end\{equation\}}

\begin{itemize}
\tightlist
\item
  if you need something, just google; e.g. ``google latex phi'' or
  ``google latex empty set'' etc.
\item
  and try it out in the jupyter notebook
\end{itemize}

\section{Second part of lecture: making sure you get the course material
and a grade for this
course}\label{second-part-of-lecture-making-sure-you-get-the-course-material-and-a-grade-for-this-course}

\begin{itemize}
\tightlist
\item
  go to the server and start a jupyter notebook
\item
  link to the server address to copy/paste in the google form
\item
  importing the course material, see chapter Schedule (this is also the
  way you will import your exam)
\item
  evaluating cells
\item
  you can choose python/R kernel
\item
  getting help: ? and TAB
\item
  code vs.~markdown cell
\item
  type some latex
\end{itemize}

\subsection{before you leave}\label{before-you-leave}

do the steps under the chapter Important

\chapter{Exam}\label{exam-1}

\section{Grade}\label{grade}

Your grade is either determined by:

\begin{itemize}
\tightlist
\item
  the midterm exam which is python only (50\%)
\item
  the exam will be R only (50\%)
\end{itemize}

or by

the resit which is based on both python and R (100\%)

Each exam lasts 3 hours. You cannot combine the resit with the midterm
etc.

\section{Useful to know}\label{useful-to-know}

The questions that we ask in the exam are based on the notebooks that we
discuss in class. Hence make sure that you have these ready before the
exam. You are allowed to use copy-paste out of these notebooks.

During the semester, you can use google to find information on
functions, error messages etc. However, during the exam you can only
access a limited number of pages.

In particular, during the exam you work in a special exam environment on
TiU computers. We have asked IT to whitelist the following websites:

\begin{itemize}
\tightlist
\item
  the russet server (where you will do your exam)
\item
  \href{gitlab.uvt.nl}{gitlab} (where you will import your exam from)
\item
  Canvas (where we will give you the command to import the exam)
\item
  \href{https://stackoverflow.com/}{stackoverflow}
\item
  \href{https://www.python.org/}{python.org}
\item
  \href{http://www.numpy.org/}{numpy}
\item
  \href{https://www.scipy.org/}{scipy}
\item
  \href{https://pandas.pydata.org/}{pandas}
\item
  \href{https://www.datacamp.com/home}{datacamp}
\item
  \href{http://www.cookbook-r.com/}{cookbook}
\item
  \href{https://dplyr.tidyverse.org/}{tidyverse.org}
\item
  \href{https://rpubs.com/}{rpubs}
\item
  \href{https://www.rstudio.com/}{rstudio}
\item
  \href{https://r4ds.had.co.nz/}{r4ds}
\item
  \href{https://www.rdocumentation.org/}{rdocumentation}
\end{itemize}

\section{Exam procedure}\label{exam-procedure}

\begin{itemize}
\tightlist
\item
  we will post the exam both on gitlab. You will get instructions how to
  get the exam on the Russet server.
\item
  finishing your exam

  \begin{itemize}
  \tightlist
  \item
    make sure that we can easily see which notebook is your exam
  \item
    that is, do not rename the exam notebook (so that we do not know
    which notebook it is)
  \item
    do not have 5 different versions of the exam notebook; we will then
    choose one at random and grade this one
  \item
    after you finish the exam, do not re-open the notebook again: we can
    see the last time the notebook was opened. If this is after you left
    the exam room, we can see this and will nullify your exam.
  \end{itemize}
\end{itemize}

\bibliography{book.bib,packages.bib}


\end{document}
